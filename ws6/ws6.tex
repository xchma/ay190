\documentclass[11pt,letterpaper]{article}

% Load some basic packages that are useful to have
% and that should be part of any LaTeX installation.
%
% be able to include figures
\usepackage{graphicx}
% get nice colors
\usepackage{xcolor}

% change default font to Palatino (looks nicer!)
\usepackage{apjfonts}
% load some useful math symbols/fonts
\usepackage{latexsym,amsfonts,amsmath,amssymb}

% comfort package to easily set margins
\usepackage[top=1in, bottom=1in, left=1in, right=1in]{geometry}

% control some spacings
%
% spacing after a paragraph
\setlength{\parskip}{.15cm}
% indentation at the top of a new paragraph
\setlength{\parindent}{0.0cm}


\begin{document}

\begin{center}
\Large
{\bf Ay190 -- Worksheet 6} \\
\large
Xiangcheng Ma \\
Date: \today
\end{center}

\section*{Discrete Fourier Transform}

(a) The function {\tt dft(x)}  that computes discrete Fourier transform is achieved in module ``{\tt dft.py}". To test this function, I uniformly generate 30 point in $[-2,2]$ for a Gaussian function
\begin{equation}
  f(x) = \frac{1}{\sqrt{2\pi}} e^{-\frac{x^2}{2}} ,
\end{equation}
and calculate the discrete Fourier transform using my function and the {\tt np.fft.fft} function. The results are listed in Table~\ref{tb}. Their results are consistent within my output accuracy, i.e., $10^{-9}$.
\begin{table}[h]
\centering
\begin{tabular}{cccc}
\hline\hline
$i$ & Size $N$ & Gauss elimination time (s) & $LU$ decomposition time (s) \\
\hline
1 & 10 & 0.00195908546448 & 0.000194072723389 \\
2 & 100 & 0.15620803833 & 0.000787019729614 \\
3 & 200 & 0.619211912155 & 0.00124096870422 \\
4 & 1000 & 18.398884058 & 0.038074016571 \\
5 & 2000 & 80.3887019157 & 0.187728881836 \\
\hline\hline
\end{tabular}
\caption{Time consumption of linear solvers.}
\label{tb}
\end{table}


(b) For 100 calculations, the accumulated time versus $N^2$ are plotted in Figure~\ref{fig1}. It shows basically a linear relation.
\begin{figure}[bth]
\centering
\includegraphics[width=0.9\textwidth]{fig1.pdf}
\caption{Computer time for {\tt dft(x)} function.}
\label{fig1}
\end{figure}

(c) For 100 calculations, the accumulated time versus $N\log(N)$ are plotted in Figure~\ref{fig2}. Although some scatter exists, the relation is in general linear.
\begin{figure}[bth]
\centering
\includegraphics[width=0.9\textwidth]{fig2.pdf}
\caption{Computer time for {\tt dft(x)} function.}
\label{fig2}
\end{figure}

\end{document}
