\documentclass[11pt,letterpaper]{article}

% Load some basic packages that are useful to have
% and that should be part of any LaTeX installation.
%
% be able to include figures
\usepackage{graphicx}
% get nice colors
\usepackage{xcolor}

% change default font to Palatino (looks nicer!)
\usepackage{apjfonts}
% load some useful math symbols/fonts
\usepackage{latexsym,amsfonts,amsmath,amssymb}

% comfort package to easily set margins
\usepackage[top=1in, bottom=1in, left=1in, right=1in]{geometry}

% control some spacings
%
% spacing after a paragraph
\setlength{\parskip}{.15cm}
% indentation at the top of a new paragraph
\setlength{\parindent}{0.0cm}


\begin{document}

\begin{center}
\Large
{\bf Ay190 -- Worksheet 7} \\
\large
Xiangcheng Ma \\
Date: \today
\end{center}

\section*{Measuring $\pi$ with an MC Experiment}
The MC estimate of $\pi$ with increasing number of points, $N$, is tabulated in Table~\ref{tb1}. The absolute error versus $\sqrt{N}$ is plotted in Figure~\ref{fig1}. The error basically follows $\sim 1/\sqrt{N}$ for not too large $N$. There is a threshold above which the error cannot be reduced by increasing $N$.
\begin{table}[h]
\centering
\begin{tabular}{ccc}
\hline\hline
$N$ & Estimated $\pi$ & error \\
\hline
1000 & 3.152 & 0.0104073464102 \\
2000 & 3.276 & 0.13440734641 \\
4000 & 3.153 & 0.0114073464102 \\
8000 & 3.129 & -0.0125926535898 \\
16000 & 3.13025 & -0.0113426535898 \\
32000 & 3.152875 & 0.0112823464102 \\
64000 & 3.136375 & -0.00521765358979 \\
128000 & 3.14140625 & -0.000186403589793 \\
256000 & 3.142421875 & 0.000829221410207 \\
512000 & 3.1400703125 & -0.00152234108979 \\
1024000 & 3.142890625 & 0.00129797141021 \\
2048000 & 3.14418945313 & 0.00259679953521 \\
4096000 & 3.14124609375 & -0.000346559839793 \\
8192000 & 3.14158642578 & -6.22780854309e-06 \\
16384000 & 3.14232861328 & 0.000735959691457 \\
32768000 & 3.14145458984 & -0.000138063746043 \\
\hline\hline
\end{tabular}
\label{tb1}
\caption{MC Measurement of $\pi$}
\end{table}


\begin{figure}[ht]
\centering
\includegraphics[width=0.5\textwidth]{fig1.pdf}
\caption{MC Estimation of $\pi$}
\label{fig1}
\end{figure}


\section*{The Birthday Paradox}
For number of people $N$, recall that there are 365 days in a year, the probability that at least two of them have same birthday is 
\begin{equation}
 P = 1 - \frac{365!/(365-N)!}{365^N} ,
\end{equation}	
where the bulk after the minus sign is just the probability that all of them have different birthdays.

WIth MC method, for each $N$, I do 20000 experiments and get an estimated probability as tabulated below.
\begin{table}[h]
\centering
\begin{tabular}{cc}
\hline\hline
$N$ & Probability \\
\hline
21 & 0.4473 \\
22 & 0.47475 \\
23 & 0.50775 \\
24 & 0.54095 \\
25 & 0.5763 \\
\hline\hline
\end{tabular}
\label{tb2}
\caption{Birthday Paradox}
\end{table}



\section*{MC Integration}
In this section, the intergration
\begin{equation}
  I = \int_2^3 f(x)~{\rm d}x = \int_2^3 x^2+1~{\rm d}x = \frac{22}{3}
\end{equation}
will be calculated using MC method.

The results are tabulated in the following table and the error as a function of $\sqrt{N}$ is plotted in Figure~\ref{fig2}. As in question 1, the error basically follows $\sim 1/\sqrt{N}$ for not too large $N$. There is a threshold above which the error cannot be reduced by increasing $N$.
\begin{figure}[ht]
\centering
\includegraphics[width=0.5\textwidth]{fig2.pdf}
\caption{MC Estimation of $\pi$}
\label{fig2}
\end{figure}

\end{document}
